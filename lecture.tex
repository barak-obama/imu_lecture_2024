
\documentclass{beamer}[10]
\usepackage{pgf}
\usepackage[danish]{babel}
\usepackage[utf8]{inputenc}
\usepackage{beamerthemesplit}
\usepackage{graphics,epsfig, subfigure}
\usepackage{url}
\usepackage{srcltx}
\usepackage{hyperref}
\usepackage{bookmark}

\definecolor{kugreen}{RGB}{50,93,61}
\definecolor{kugreenlys}{RGB}{132,158,139}
\definecolor{kugreenlyslys}{RGB}{173,190,177}
\definecolor{kugreenlyslyslys}{RGB}{214,223,216}
\setbeamercovered{invisible}
\mode<presentation>
\usetheme[numbers,totalnumber,compress,sidebarshades]{PaloAlto}
\setbeamertemplate{footline}[frame number]

  \usecolortheme[named=kugreen]{structure}
  \useinnertheme{circles}
  \usefonttheme[onlymath]{serif}
  \setbeamercovered{invisible}
  \setbeamertemplate{blocks}[rounded][shadow=true]

\logo{\includegraphics[width=0.8cm]{KULogo}}
%\useoutertheme{infolines} 
\title{Equations in Hyperbolic(-esque) Groups}
\author{Barak Ohana}
\institute{The Hebrew University of Jerusalem}
\date{9 Sep 2024}

\newcommand{\sol}{\text{V}_{\Gamma}}



\begin{document}
\frame{\titlepage \vspace{-0.5cm}}

% \frame
% {
% \frametitle{Overview}
% \tableofcontents%[pausesection]
% }

\section{First section}

\frame{
    \frametitle{Equations in Groups}
Let $\Gamma$  be a group\pause, and let $\Sigma\left(x_{1},\ldots,x_{d}\right)$ a system of equation on d variables. \pause
        We define the \textbf{set of solution of $\Sigma$ in $\Gamma$} to be the set
            \begin{equation*}
                \sol\left(\Sigma\right)=\left\{ \left(g_{1},\ldots,g_{d}\right)\in\Gamma^{d}\mid\Sigma\left(g_{1},\ldots,g_{d}\right)=_{\Gamma}1\right\} 
            \end{equation*}

            \pause

}


\subsection{Sample subsection}

\frame{
\frametitle{Sample Frame Title No. 2}
\begin{itemize}
\item First item
\item Second item
\item Third item
\end{itemize}
}

\section{Second section}

\frame{
\frametitle{Sample Frame Title No. 3}
Lorem ipsum dolor sit amet, consectetur adipiscing elit, sed do eiusmod tempor incididunt ut labore et dolore magna aliqua.
\begin{block}{Something important}
Einstein's formula
$$E=mc^2$$
\end{block}
}


\end{document}